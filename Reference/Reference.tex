\documentclass[paper={550pt,2910pt},lualatex , ja=standard]{bxjsreport}
\usepackage{luatexja-otf}
\usepackage[deluxe]{luatexja-preset}
\usepackage{url}
\usepackage{listings}
\usepackage{hyperref}
\usepackage{xcolor}
\usepackage{tcolorbox}
\usepackage{tikz}
\usepackage{bxpapersize}
\usepackage{layout}
\usepackage{geometry}
\usepackage{siunitx}

\setlength{\hoffset}{-1in+50pt}
\setlength{\voffset}{-1in+50pt}
\setlength{\textwidth}{450pt}
\setlength{\topmargin}{0pt}
\setlength{\headheight}{0pt}
\setlength{\headsep}{0pt}
\setlength{\textheight}{2810pt}
\tcbuselibrary{listings}
\hypersetup{%
luatex,
pdfencoding=auto,
colorlinks=true,%
allcolors=blue
}

\definecolor{White}{rgb}{1,1,1}
\definecolor{Orange}{rgb}{0.9,0.1,0}
\lstset{
  basicstyle={\ttfamily\scriptsize}, % 使用フォント
  classoffset=1,
  breaklines=true,
  identifierstyle={},
  commentstyle={},
  keywordstyle={\bfseries},
  ndkeywordstyle={},
  stringstyle={\ttfamily},
  columns=fixed,
  basewidth=0.5em,
  numberstyle={\tiny},
  stepnumber=1,
  tabsize=4,
  keywordstyle={\color{blue}}, %キーワード(int, ifなど)の書体指定
  commentstyle={\color{OliveGreen}}, %注釈の書体
  stringstyle={\color{Orange}}, %文字列
  showstringspaces=false,  %文字列中の半角スペースを表示させない
  keepspaces=true,
  rulesep = 1pt,
  xrightmargin=0zw,                     % 右マージンのサイズ.
  xleftmargin=1.6zw,                    % 左マージンのサイズ.行番号が2桁でも行左端からはみ出ない値.
}

\makeatletter
\def\lst@lettertrue{\let\lst@ifletter\iffalse}
\makeatother

\pagestyle{empty}
\begin{document}

\def\xmlcol{-20pt}
\def\xmlrowVal{0pt}
\def\xmlrowValTL{10pt}
\def\xmlrowDef{150pt}
\def\xmlrowExp{200pt}

\definecolor{backcolor}{rgb}{0.9,0.9,0.9}
\definecolor{framecolor}{rgb}{0.8,0.8,0.8}
\definecolor{xsdbackcolor}{rgb}{0.9,0.9,1}
\definecolor{xsdframecolor}{rgb}{0.7,0.7,0.9}
\definecolor{White}{rgb}{1,1,1}

\newcommand{\mktree}[3]{
    \node[anchor=west,font=\fontsize{\xmlrowValTL}{\xmlrowValTL}\selectfont] at(\xmlrowVal+\xmlrowValTL*#2,\xmlcol*#3){#1};
    \ifnum #2=0
    \else
    \draw (\xmlrowVal+\xmlrowValTL*#2-\xmlrowValTL/4,\xmlcol*#3)--(\xmlrowVal+\xmlrowValTL*#2+\xmlrowValTL/4,\xmlcol*#3);
    \fi
    }
\newcommand{\drawtreeline}[3]{
    \draw (\xmlrowVal+\xmlrowValTL*#1-\xmlrowValTL/4,\xmlcol*#2+\xmlcol/2)--(\xmlrowVal+\xmlrowValTL*#1-\xmlrowValTL/4,\xmlcol*#3);
    }
\def\postbreak{}
%\def\postbreak{\mbox{\textcolor{red}{$\hookrightarrow$}\space}}
\newtcblisting{reflisting}[2][]{
      arc=5pt,
      top=10pt,
      bottom=10pt,
      left=0pt,
      right=0pt,
      boxrule=1pt,
      colback=backcolor,
      colframe=framecolor,
      listing only,
      hbox,
      #1,
      listing options={
        breaklines=true,
        postbreak=\postbreak,
        #2
      }
}
\newtcblisting{refxsdlisting}[1][]{
      arc=5pt,
      top=10pt,
      bottom=10pt,
      left=15pt,
      right=0pt,
      boxrule=1pt,
      colback=xsdbackcolor,
      colframe=xsdframecolor,
      listing only,
      hbox,
      #1,
      coltitle=black,
      listing options={
        language=XML,
        breaklines=true,
        postbreak=\postbreak,
        escapechar=!,
      }
}

%\layout
\chapter*{ExtendedTrainSchedulerWithOudia}

\section*{構文解説}

\subsection*{directionについて}

上り線の列車は\texttt{InBound}の文字列 , 下り線の列車は\texttt{OutBound}の文字列 とします.

\subsection*{主関数}
各ステートメントにつき1つのみ記述できます. 各ステートメントの最後にのみセミコロン ; を記述してください.
\subsubsection*{\texttt{Stop(station, car)}}
\begin{reflisting}[]{language=C++,escapechar=\@,}
BveEx.User.Toukaitetudou.ExtendedTrainSchedulerWithOudia.Track[string direction].Stop(sring station,int car);
\end{reflisting}
進行方向 direction の向きで両数 car 以下の列車の駅 station における停車位置をその距離程に設定します.

同一駅・同一列車に対し有効な設定が複数存在する場合, car の値が最も小さいもの 距離程の値が最も小さいものの順で優先されます.

オプション関数Series, IgnoreSeriesを利用できます.

\subsubsection*{\texttt{ToHere(v, a)}}
\begin{reflisting}[]{language=C,escapechar=\@,}
BveEx.User.Toukaitetudou.ExtendedTrainSchedulerWithOudia.Track[string direction].Accelerate.ToHere(double v,double a);
\end{reflisting}
進行方向 direction の向きの列車の加減速を 加速度 a $[\si{\kilo\meter/\hour/\second}]$, 目標速度 v $[\si{\kilo\meter/\hour}]$ で, その距離程 "まで" で完了するように行います.

オプション関数Series, IgnoreSeries, Stop, IgnoreStop, Pass, IgnorePassを利用できます.
\subsubsection*{\texttt{FromHere(v, a)}}
\begin{reflisting}[]{language=C,escapechar=\@,}
BveEx.User.Toukaitetudou.ExtendedTrainSchedulerWithOudia.Track[string direction].Accelerate.FromHere(double v,double a);
\end{reflisting}
    進行方向 direction の向きの列車の加減速を 加速度 a $[\si{\kilo\meter/\hour/\second}]$, 目標速度 v $[\si{\kilo\meter/\hour}]$ で, その距離程 "から" 開始します.
    
    オプション関数Series, IgnoreSeries, Stop, IgnoreStop, Pass, IgnorePassを利用できます.
\subsection*{オプション関数}
主関数に続けて記述することで対象列車の絞り込みをします. 
1つの主関数に対し複数のオプション関数を指定できます. 
主関数によって利用できるオプション関数は異なります.
本項において~は絞り込みするステートメントを表すものとします.
\subsubsection*{Series(series)}
\begin{reflisting}[]{language=C,escapechar=\@,}
~.Series(string series,...);
\end{reflisting}
対象列車を車種が series... であるものに絞り込みます.
seriesは複数回記述できます.
\subsubsection*{IgnoreSeries(ignoreseries)}
\begin{reflisting}[]{language=C,escapechar=\@,}
~.IgnoreSeries(string ignoreseries,...);
\end{reflisting}
対象列車を車種が ignoreseries... でないものに絞り込みます.
ignoreseriesは複数回記述できます.
\subsubsection*{Stop(stops)}
\begin{reflisting}[]{language=C,escapechar=\@,}
~.Stop(string stops,...);
\end{reflisting}
対象列車を stops... に停車するものに絞り込みます.
stopsは複数回記述できます.
\subsubsection*{IgnoreStop(ignorestops)}
\begin{reflisting}[]{language=C,escapechar=\@,}
~.IgnoreStop(string ignorestops,...);
\end{reflisting}
対象列車を ignorestops... に停車しないものに絞り込みます.
ignorestopsは複数回記述できます.
\subsubsection*{Pass(passes)}
\begin{reflisting}[]{language=C,escapechar=\@,}
~.Pass(string passes,...);
\end{reflisting}
対象列車を passes... を通過するものに絞り込みます.
passesは複数回記述できます.
\subsubsection*{IgnorePass(ignorepasses)}
\begin{reflisting}[]{language=C,escapechar=\@,}
~.IgnorePass(string ignorepasses,...);
\end{reflisting}
対象列車を ignorepasses... を通過しないものに絞り込みます.
ignorepassesは複数回記述できます.

\section*{設定ファイル解説}

ExtendedTrainSchedulerWithOudia.dllと同階層のExtendedTrainSchedulerWithOudia.Config.xmlに記述します

\begin{tikzpicture}
    \mktree{要素・属性}{0}{0}
    \mktree{ExtendedTrainSchedulerWithOudiaConfig}{0}{1}
    \mktree{Config}{1}{2}
    \mktree{RealTimeUpdate}{2}{3}
    \mktree{UpdateOutputFile}{2}{4}
    \mktree{VehicleDirection}{2}{5}
    \mktree{VehicleOperationNumber}{2}{6}
    \mktree{Oud2FilePath}{2}{7}
    \mktree{TimeTableName}{2}{8}
    \drawtreeline{2}{2}{8}
    \mktree{Operation}{1}{9}
    \mktree{OperationNumber}{2}{10}
    \mktree{InBoundTrainFilePath}{2}{11}
    \mktree{OutBoundTrainFilepath}{2}{12}
    \mktree{Series}{2}{13}
    \mktree{Cars}{2}{14}
    \drawtreeline{2}{9}{14}
    \drawtreeline{1}{1}{9}
\end{tikzpicture}
\subsection*{ExtendedTrainSchedulerWithOudiaConfig}
\begin{refxsdlisting}[title=XSD定義,]
<xs:element name="ExtendedTrainSchedulerWithOudiaConfig">
    <xs:complexType>
        <xs:sequence>
            <xs:element name="Config" type="Config"/>
            <xs:element name="Operation" type="Operation" minOccurs="0" maxOccurs="unbounded"/>
        </xs:sequence>
    </xs:complexType>
    <xs:unique name="OperationNumber" id="OperationNumber">
        <xs:selector xpath="Content"/>
        <xs:field xpath="Config/@VehicleOperationNumber" />
        <xs:field xpath="Operation/@OperationNumber" />
    </xs:unique>
</xs:element>
\end{refxsdlisting}
(要素) ルート要素です. 1個のConfig要素と0個以上のOperation要素を持ちます.
\subsection*{Config}
\begin{refxsdlisting}[title=XSD定義,]
<xs:complexType name="Config">
    <xs:attribute name="RealTimeUpdate" type="xs:boolean" default="true"/>
    <xs:attribute name="UpdateOutputFile" type="xs:boolean" default="true"/>
    <xs:attribute name="VehicleDirection" type="Bound" use="required"/>
    <xs:attribute name="VehicleOperationNumber" type="xs:NMTOKEN" use="required"/>
    <xs:attribute name="Oud2FilePath" type="xs:anyURI" use="required"/>
    <xs:attribute name="TimeTableName" type="xs:NMTOKEN" use="required"/>
</xs:complexType>
\end{refxsdlisting}
(要素) 自列車やダイヤ全体に係ることの設定をします. 0個の子要素を持ちます.
\subsubsection*{RealTimeUpdate}
(属性) trueの場合プラグインが生成した構文を読み込みます. 省略可能です. 省略した場合, 値はtrueであるものとして扱われます.
\subsubsection*{UpdateOutputFile}
(属性) trueの場合生成した構文をファイルに書き出します. 省略可能です. 省略した場合, 値はtrueであるものとして扱われます.
\subsubsection{VehicleDirection}
(属性) 自列車の direction を記述します. 省略できません.
\subsubsection*{VehicleOperationNumber}
(属性) 自列車の運用番号を記述します. 省略できません.
\subsubsection*{Oud2FilePath}
(属性) ダイヤを読み込むOudiaのファイルへのパスを記述します. 省略できません.
\subsubsection*{TimeTableName}
(属性) 読み込むダイヤの名前を記述します. 省略できません.
\subsection*{Operation}
\begin{refxsdlisting}[title=XSD定義,]
<xs:complexType name="Operation">
    <xs:attribute name="OperationNumber" type="xs:NMTOKEN" use="required"/>
    <xs:attribute name="InBoundTrainFilePath" type="xs:anyURI" use="required"/>
    <xs:attribute name="OutBoundTrainFilepath" type="xs:anyURI" use="optional"/>
    <xs:attribute name="Series" type="xs:NMTOKEN" use="optional"/>
    <xs:attribute name="Cars" type="xs:positiveInteger" use="required"/>
</xs:complexType>
\end{refxsdlisting}
(要素) 運用番号と車両情報の紐づけをします. 0個の子要素を持ちます.
\subsubsection*{OperationNumber}
(属性) 紐づける運用番号を指定します. 値はConfig要素のVehicleOperationNumber属性の値, 他のOperation要素のOperationNumberの値と重複できず, 省略できません.
\subsubsection*{InBoundTrainFilePath}
(属性) 上り列車として運転するときの他列車ファイルを指定します. 省略できません.
\subsubsection*{OutBoundTrainFilepath}
(属性) 下り列車として運転するときの他列車ファイルを指定します. 省略可能です. 省略した場合, 値はInBoundTrainFilePath属性の値であるものとして扱います.
\subsubsection*{Series}
(属性) 車種を指定します. 省略可能です.
\subsubsection*{Cars}
(属性) 両数を指定します. 省略できません.

\end{document}